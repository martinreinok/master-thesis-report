%-------------------------------------------------------------------------------------------------
% This settings.tex contains settings required for *all* documents (reports, presentations, etc)
% Project or Report specific settings should go to their own settings files (eg CE/settings.tex)
% This file is included after the class definition and before project and report specific settings 
%-------------------------------------------------------------------------------------------------

%--------Useful packages (required by the example files, turn off if you do not use them)-------
\usepackage{babel}					% Add language specific support
%\usepackage{makeidx}				% Index support
%\usepackage[totoc,justific=RaggedRight]{idxlayout}	% Make last page of index balanced and add index to toc
\usepackage{booktabs}               % Make nice tables
\usepackage{caption}				% Provides means to style captions
%\usepackage{subcaption}				% Provides support for (sub)figures and (sub)tables
%\usepackage{float}					% Improved interface for floating objects (eg figures, tables, ...)
\usepackage{enumitem}				% Add styling support to (enumerate) environments
\usepackage{listings}				% Allows (external) source files to be shown in a syntax highlighted way
%\usepackage{amsmath}				% Provides miscellaneous enhancements for documents containing formulas
\usepackage{datetime}				% Provides commands for displaying the current time
%\usepackage{etoolbox}				% Provides \AtBeginEnvironment command
\usepackage{eurosym}				% Defines \euro command to display euro symbols
%\usepackage{appendix}				% Makes it possible to modify appendix numbering
%\usepackage{longtable}				% Allows tables to span multiple pages
%\usepackage{units}					% Shows units (eg m/s) in a nice way
%\usepackage{ctable}				% Provides \ctable command for the typesetting of table and figure floats
%\usepackage{ccaption}				% Support continuation captions (eg multi-page tables)
%\usepackage{verbatim}				% Adds verbatim environment, in which texts are exactly copied to the output
%\usepackage{pdfpages}				% Include PDF pages/documents in the current document
\usepackage{../ram-report/requirements}

\iffinalversion
	\usepackage[final]{../ram-report/notes}% Add note commands, [final] removes all notes from the document
	\usepackage[final]{../ram-report/rro}  % Add Rich Report Outline support, [final] removes all RRO output from document
\else
	\usepackage{../ram-report/notes}       % Add note commands
	\usepackage{../ram-report/rro}         % Add Rich Report Outline support
\fi


%% Spacing possibilities for captions are available as well
% See captions.pdf for all options!
\captionsetup{font=small,labelfont=bf}

%% Center all figures by default
%% http://tex.stackexchange.com/questions/2651/should-i-use-center-or-centering-for-figures-and-tables
\makeatletter
\g@addto@macro\@floatboxreset\centering
\makeatother

%% Make use small font size in verbatim environment
% Note: AtBeginEnvironment is provided by etoolbox package
%\AtBeginEnvironment{verbatim}{\small}

%% Include verbatim in the subfigure env
% From: subfig.pdf, section 4.4
% <Uncomment if verbatim is required in subfloat>:
%\makeatletter
%\newbox\sf@box
%\let\orig@subfloat\subfloat
%\renewenvironment{subfloat}[2][]%
%{ \def\sf@one{#1}%
%  \def\sf@two{#2}%
%  \setbox\sf@box\hbox
%  \bgroup}%
%{ \egroup
%  \ifx\@empty\sf@two\@empty\relax
%    \def\sf@two{\@empty}
%  \fi
%  \ifx\@empty\sf@one\@empty\relax
%    \orig@subfloat[\sf@two]{\box\sf@box}%
%  \else
%    \orig@subfloat[\sf@one][\sf@two]{\box\sf@box}%
%  \fi}
%\makeatother
%% Uncomment till here
  
%% Automatically provide H option for floats
% Requires float package
% \floatplacement{figure}{H} 
% \floatplacement{table}{H} 

%% abbreviation making
\newcommand{\abbr}[1]{(\textit{#1})}

%lstlisting settings
\lstset{	
		aboveskip=0.33\baselineskip,%
		belowskip=0.33\baselineskip,
		numbers=none, %no line numbers
		language=,% No (default) code language in the document
		xleftmargin=1em,
		framexleftmargin=0.5em,
		xrightmargin=1em,
		framexrightmargin=0.5em,
		tabsize=3,
		morecomment=[s][\itshape]{<}{>}, %also define <> as comment
		morecomment=[s][\itshape]{[}{]}, %also define [] as comment
		basicstyle=\ttfamily, % mono-space font
		showstringspaces=false, % Don't put 'underscores' ,_, in strings (it's ugly)
		keepspaces=true, % Preserve whitespaces (by default it can drop spaces `to fix column alignment')
		breaklines=true,
		breakatwhitespace=true
}

%lstinline with empty language definition
\lstdefinelanguage{empty}{}
\newcommand{\mylstinline}[1]{{\lstinline[language=empty]{#1}}}

% Define additional parameters for floating style. Floating listings can have a caption and label (similar
% to figures, but with a different syntax):
% \begin{lstfloat}[caption={The caption}, label={lst:thelabel}]
%	<the code>
% \end{lstfloat}	
\lstdefinestyle{lstfloatstyle}{
	float=tp, % Position on top or 
	captionpos=b,%
	%  numbers=left, %show line numbers  - Uncomment if you want numbers by default
	numberstyle=\tiny,%
	numbersep=5pt,%
	frame={tb}, % line on top and bottom, like a table (see booktabs package)
	framerule=0.77pt, % Thickness of top and bottom line (same has \toprule of booktabs)
	xleftmargin=0pt, xrightmargin=0pt,
	framexleftmargin=0pt, framexrightmargin=0pt
}
\lstnewenvironment{lstfloat}[1][]{\lstset{style=lstfloatstyle, #1}}{}

% Compact itemize, enumerate and description (uses enumitem package)
\setlist{topsep=-4pt,noitemsep}

%% Don't show warnings like: ``PDF inclusion: found PDF version <1.x>, but at most version <1.4> allowed
% Uncomment if you experience these kind of warnings 
%\ifpdf
%	\pdfminorversion=6 
%\fi