\chapter{Introduction}
\label{ch:intro}
Cardiovascular diseases remain as a predominant cause of global mortality, motivating the exploration of alternative medical interventions to open surgery, which has higher odds of mortality and complications. An viable alternative has been endovascular operation, which primary benefit is faster recovery due to the minimally invasive approach. 

Normally, the endovascular operation is performed with the help of Computed Tomography (CT) imaging, which involves the use of X-rays - a form of ionizing radiation that increases the risk of cancer. 

% Introduce X-ray CT, CT Fusion and eventually why MRI could be better.

An alternative to CT is the use of Magnetic Resonance Imaging (MRI) technology, which does not involve any ionizing radiation and has the potential for better quality imaging compared to CT. The CathBot project proposes a MR-safe teleoperation platform to manipulate endovascular devices remotely and to provide operators of the CathBot with a feedback loop for endovascular tasks. 

This thesis focuses on addressing the shortcomings of MRI for endovascular operations by implementing a Real-Time passive guidewire detection for MRI, and integrating detection with CathBot to create a feedback loop and safety layers to ensure patient protection during operation. The safety layers will be integrated on the CathBot using haptic feedback capabilities, which inform the operator of potential issues such as collisions with vessels.